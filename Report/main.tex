\documentclass[aspectratio=54,xcolor=dvipsnames]{beamer}
\usetheme{SimpleDarkBlue}

% Packages
\usepackage[utf8]{inputenc}
\usepackage{amsmath, amssymb}
\usepackage{graphicx}
\usepackage{hyperref}
\usepackage{physics}
% \usepackage{siunitx}
% \AtBeginDocument{\RenewCommandCopy\qty\SI}


% Title Information
\title{Computational Electromagnetics Project}
\author{Francesco Fischetti}
\institute{Politecnico di Milano}
\date{\today}

\begin{document}

% Title Slide
\begin{frame}
    \titlepage
\end{frame}

% Outline Slide
\begin{frame}{Outline}
    \tableofcontents
\end{frame}

\section{Introduction}
\begin{frame}{Introduction}
    \begin{itemize}
        \item Motivation for computational electromagnetics
        \item Overview of the project objectives
    \end{itemize}
\end{frame}

\section{Theory}
\begin{frame}{Maxwell's Equations (Differential Form)}
    Maxwell's equations in differential form are given by:
    \begin{align*}
        \nabla \cdot \vec{D} &= \rho \\
        \nabla \cdot \vec{B} &= 0 \\
        \nabla \times \vec{E} &= -\frac{\partial \vec{B}}{\partial t} \\
        \nabla \times \vec{H} &= \vec{J} + \frac{\partial \vec{D}}{\partial t}
    \end{align*}

    \begin{center}
    %\scriptsize
    \begin{tabular}{|c|l|c|}
        \hline
        Symbol & Quantity & Unit \\
        \hline
        $\vec{D}(\vec{r})$ & Electric flux density & $C/m^2$ \\
        $\vec{B}(\vec{r})$ & Magnetic flux density & $T$ \\
        $\vec{E}(\vec{r})$ & Electric field & $V/m$ \\
        $\vec{H}(\vec{r})$ & Magnetic field & $A/m$ \\
        $\vec{J}(\vec{r})$ & Current density & $A/m^2$ \\
        $\rho(\vec{r})$ & Volume charge density & $C/m^3$ \\
        \hline
    \end{tabular}
    %\normalsize
    \end{center}
\end{frame}

\begin{frame}{Constitutive Relations}
    For a linear and isotropic medium the constitutive relations read
    \begin{align*}
        \vec{D} &= \epsilon \vec{E} \\
        \vec{B} &= \mu \vec{H} \\
        \vec{J} &= \sigma \vec{E}
    \end{align*}
    \vspace{0.5em}
    where
    \begin{center}
    \begin{tabular}{|c|l|c|}
        \hline
        Symbol & Quantity & Unit \\
        \hline
        $\epsilon$ & Dielectric permittivity & $[F/m]$ \\
        $\mu$ & Magnetic permeability & $[H/m]$ \\
        $\sigma$ & Electric conductivity & $[S/m]$ \\
        \hline
    \end{tabular}
    \end{center}
\end{frame}

\begin{frame}{Maxwell's Equations}
    \begin{align*}
        \nabla \cdot \vec{E} &= \frac{\rho}{\varepsilon_0} \\
        \nabla \cdot \vec{B} &= 0 \\
        \nabla \times \vec{E} &= -\frac{\partial \vec{B}}{\partial t} \\
        \nabla \times \vec{B} &= \mu_0 \vec{J} + \mu_0 \varepsilon_0 \frac{\partial \vec{E}}{\partial t}
    \end{align*}
\end{frame}

\begin{frame}{Numerical Methods}
    \begin{itemize}
        \item Finite Difference Time Domain (FDTD)
        \item Finite Element Method (FEM)
        \item Method of Moments (MoM)
    \end{itemize}
\end{frame}

\section{Implementation}
\begin{frame}{Simulation Setup}
    \begin{itemize}
        \item Domain discretization
        \item Boundary conditions
        \item Source excitation
    \end{itemize}
\end{frame}

\begin{frame}{Algorithm Flowchart}
    \begin{center}
        % \includegraphics[width=0.7\textwidth]{flowchart.png}
        \textit{Flowchart image not available.}
    \end{center}
\end{frame}

\section{Results}
\begin{frame}{Simulation Results}
    \begin{itemize}
        \item Field distributions
        \item Convergence analysis
        \item Validation with analytical solutions
    \end{itemize}
    \begin{center}
        % \includegraphics[width=0.6\textwidth]{results.png}
        \textit{Results image not available.}
    \end{center}
\end{frame}

\section{Conclusion}
\begin{frame}{Conclusion}
    \begin{itemize}
        \item Summary of findings
        \item Future work
    \end{itemize}
\end{frame}

% References
\begin{frame}{References}
    \footnotesize
    \begin{thebibliography}{99}
        \bibitem{sadiku} M. N. O. Sadiku, \emph{Numerical Techniques in Electromagnetics}, CRC Press, 2000.
        \bibitem{taflove} A. Taflove, S. C. Hagness, \emph{Computational Electrodynamics: The Finite-Difference Time-Domain Method}, Artech House, 2005.
    \end{thebibliography}
\end{frame}

\end{document}